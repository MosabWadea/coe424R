%&latex
%%----------------------------------------------------------------------
%% ieeepes_skel.tex
%%
%% Skeleton file for papers for the IEEE Power Engineering Society using
%% package ieeepes.
%%
%% Not copyrighted. Copy this file to a different name and fill in your
%% text.
%%
%% Volker Kuhlmann
%% c/o EEE Dept
%% University of Canterbury
%% Private Bag 4800
%% Christchurch, New Zealand
%% Email: KUHLMAV@ELEC.CANTERBURY.AC.NZ
%%
% 1.3  13Apr99  Updated for ieeepes 4.0.
% 1.2  16Nov95  Fixed discussion, closure. Added summary.
% 1.1  12Nov95  Finished first release.
% 1.02 09Nov95  Option PStimes.
% 1.0  07Nov95  Created.
%%----------------------------------------------------------------------
 
\documentclass[a4paper,twoside]{article}


\usepackage[%
               psphotos,      % uncomment those options you want
               photofit,%
        %       PStimes,%
        ]{ieeepes}



\title{Bus Tracking and Monitoring System Using RFID Technology}

\author{
        Mosab Wadea\\
        201021320
\and
        Omar Amin\\
        201073280
\and
        Ahmad Bajobair\\
        201152850
\and
        Mahdi Sahel\\
        201152070
}


\usepackage{graphicx}

\begin{document}

\maketitle


\begin{abstract}
Put the text of your abstract here.
\end{abstract}


%beginning sections
\section{Introduction}



\section{Problem}
The main transportation method in KFUPM is the bus system that is supervised and ran by the transportation department. The problem with the system depends on the knowledge of the student about the timing of the busses movements and when busses are located in any station. If a small delay or error happens in the system it will affect all the time schedule and the flow of the busses. This error is most likely to occur and can not be prevented. The problem with the students is that they will not be able to identify when a delay occurs and wither a bus is available or not.
%

\section{Solution}
The solution proposed in this project is to provide a monitoring system for both the student and the administrators. Where students can access a website to check for busses and their availability, as will as the administrators who can view the tracking results and evaluate the efficiency of the system.

\subsection{Requirements}
In order for the project to fulfill the needs it must sustain the following requirements:
\begin{itemize}
\item
Identify the busses and their assigned lines.
\item
Detect the busses and their movements.
\item
Detect busses on the run without the need to stop.
\item
Users can interact with the system and submit complaints.
\item
Administrators can access the system to evaluate the efficiency.
\end{itemize}

\subsection{Specifications}
In order to make these requirements feasible the project has to implement the following specifications:
\begin{itemize}
\item 
Attach RFID passive tag to each bus.
\item
Install RFID reader and antenna at each bus station.
\item
Develop web based application that connects to the readers and fitch data and displays them.
\end{itemize}
%

\section{Project Design}

\subsection{Block diagram}

\subsection{Software architecture}



\section{Methodology}

\subsection{Hardware}
%add subsections here to place the test results

\subsection{Software}
%add subsections here to describe code



\section{Discussion}

\subsection{Challenges}

\subsection{limitations}



\section{Conclusion}






%this is the figure codes

% \begin{figure}
% \centering
% \fbox{\includegraphics{c:/users/mosab/desktop/capture.png}}
% \caption{This is the caption for figure \#1.}
% \label{figure1}
% \end{figure}

%here is the table code
 
% \begin{table}
% \caption{This is the caption for table \#1. Make sure it goes
% \emph{before} the table!}
% \label{table1}
% \centering
% table matter
% \end{table}

%here is the refrencing codes

% Figure and table references: \figref{figure1}, \tabref{table1}. Use
% these at the beginning and within a sentence.




\end{document}

